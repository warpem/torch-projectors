\documentclass[10pt,a4paper]{article}

% Essential packages
\usepackage[utf8]{inputenc}
\usepackage[T1]{fontenc}
\usepackage{amsmath,amsfonts,amssymb}
\usepackage{graphicx}
\usepackage{booktabs}
\usepackage{multirow}
\usepackage{array}
\usepackage{url}
\usepackage{hyperref}
\usepackage{natbib}
\usepackage{geometry}
\usepackage{algorithm}
\usepackage{algpseudocode}
\usepackage{listings}
\usepackage{xcolor}

% Page geometry
\geometry{left=1.5cm, right=1.5cm, top=2.5cm, bottom=2.5cm}

% Paragraph formatting - no indentation, vertical spacing between paragraphs
\setlength{\parindent}{0pt}
\setlength{\parskip}{0.3\baselineskip}

% Hyperref setup
\hypersetup{
    colorlinks=true,
    linkcolor=blue,
    filecolor=magenta,      
    urlcolor=cyan,
    citecolor=blue
}

% Code listing setup
\lstset{
    basicstyle=\footnotesize\ttfamily,
    commentstyle=\color{gray},
    keywordstyle=\color{blue},
    stringstyle=\color{red},
    breaklines=true,
    showstringspaces=false,
    frame=single,
    language=Python
}

\title{torch-projectors: A High-Performance Differentiable Projection Library for PyTorch}

\author{
    Dimitry Tegunov \\
    \small{tegunovd@gene.com}
}

\date{\today}

\begin{document}

\maketitle

\begin{abstract}
Fourier-space projection operations are central to many algorithms in electron microscopy single-particle analysis and electron tomography. Methods employing machine learning require these operations to be differentiable, enabling end-to-end model training. While an implementation using solely PyTorch's built-in operations is possible, it is often too slow for practical use. This paper introduces \texttt{torch-projectors}: a high-performance, cross-platform library for differentiable Fourier-space projection operations in PyTorch. The library implements 2D and 3D forward and backward projection operators with support for linear and cubic interpolation, and gradient calculation for all inputs. Designed to be efficient on CPU, Apple Silicon (MPS), and CUDA devices, \texttt{torch-projectors} outperforms the existing Python-only \texttt{torch-fourier-slice} library by 1–2 orders of magnitude in many common scenarios.
\end{abstract}

\section{Introduction}
Transmission electron microscopes (TEMs) capture 2D projections of the sample's Coulomb potential at very high optical resolution. Given multiple projections from different angles, the original 3D potential can be reconstructed computationally, enabling atomic model building in favorable cases. Over the past decade, cryogenic electron microscopy (cryo-EM) has emerged as the dominant method for solving protein structures and investigating their dynamics. If the sample is thin enough for the thin-object approximation to hold, the Fourier slice theorem states that each 2D projection corresponds to a central slice through the 3D Fourier transform of the potential. This relationship is fundamental to many algorithms in electron microscopy (EM) single-particle analysis (SPA) and electron tomography (ET).

Modern machine learning methods can quickly optimize vast amounts of parameters leveraging gradient back-propagation in differentiable models. PyTorch [PyTorch ref] is a popular framework for implementing such models, providing automatic differentiation capabilities. However, many EM algorithms require differentiable Fourier-space projection operations that are not natively supported by PyTorch. While it is possible to implement these operations using PyTorch's built-in functions [torch-fourier-slice ref], the performance is often insufficient, and the memory overhead is too high for practical use. \texttt{torch-projectors} implements each projection operation in a single call without the need to store intermediate values. Thus, the memory footprint is limited to the input and output tensors, making the operators suitable for large-scale models.

RELION [methods in enzymology ref] has popularized an accurate, yet computationally efficient interpolation method for Fourier-space projection operations in single-particle analysis (SPA). Forward projections sample from a reference that has been oversampled by zero-padding it in real space, while backward projections insert data into a similarly oversampled reconstruction, and the reconstruction is cropped to its regular size in real space afterwards. This approach improves the accuracy of computationally cheap linear interpolation at the cost of increased memory usage. While it works well for conventional SPA algorithms, novel methods utilizing machine learning may be memory-constrained, or have very short-lived intermediate data that need costly Fourier transforms in every forward and backward pass to enable oversampling. \texttt{torch-projectors} implements Catmull-Rom cubic interpolation [Catmull-Rom spline ref] in addition to linear, which can significantly improve accuracy without oversampling.

Another elegant approach implemented in RELION is the use of two volumes for reconstructions: one volume with complex-valued data, and another volume to keep track of real-valued weights for each Fourier component. This allows for efficient normalization of per-component sampling counts as well as Wiener-like deconvolution approaches when the weights incorporate the contrast transfer function (CTF) of the back-projected images. \texttt{torch-projectors} implements this approach, and enables gradient back-propagation through the weight components if desired.

While 3D->2D forward projection and 2D->3D backward projection are the dominant operations in SPA, procedures like 2D particle classification require projections in both directions without a change in dimensionality. All four operations are implemented in \texttt{torch-projectors}.

\section{Methods}

\subsection{Data Conventions}

\texttt{torch-projectors} operates exclusively in Fourier space using PyTorch's RFFT format, following FFTW conventions. Unlike RELION's implementation, tensors on both ends of the projection operations remain RFFT-formatted, without intermediate stages with different data layouts. 

For rotations to work correctly, reconstruction and projection data must be shifted (fftshift) in real space to place their center at the 0th tensor element before FFT. Results must be inverse-shifted (ifftshift) in real space to restore conventional centering.

All spatial dimensions must be square (2D) or cubic (3D) with even sizes. 2D reconstructions use shape $[B, N, N/2+1]$ where $B$ is the batch dimension. 3D reconstructions use shape $[B, D, H, W/2+1]$. Projections follow the same RFFT convention with shape $[B, P, N, N/2+1]$, where $B$ is the reconstruction batch dimension, and $P$ is the pose dimension.

Rotations are specified as matrices: $[B, P, 2, 2]$ for 2D and $[B, P, 3, 3]$ for 3D. Translation shifts use shape $[B, P, 2]$ and are applied via phase modulation. A positive shift value moves the image contents along the positive direction of the respective axis. In back-projection, the shift is applied in the opposite direction to ensure it is the inverse of the forward projection with the same parameters. Batch broadcasting is supported where $B$ can be 1 or match the reconstruction batch size $B$; or the $P$ dimension in either rotations or shifts (but not both) can be 1 to use the same value over all poses. This enables efficient processing of multiple reconstructions with shared or individual poses.

\subsection{Interpolation}

\texttt{torch-projectors} supports linear and cubic interpolation methods for sampling Fourier-space data at non-integer coordinates. Linear interpolation uses standard multilinear kernels: bilinear for 2D operations (4-point support) and trilinear for 3D operations (8-point support).

Cubic interpolation employs separable Catmull-Rom kernels with parameter $a = -0.5$, providing $C^1$ continuity and exact interpolation through control points. The kernel function is defined as:

\begin{equation}
w(s) = \begin{cases}
(a + 2)|s|^3 - (a + 3)|s|^2 + 1 & \text{if } |s| \leq 1 \\
a|s|^3 - 5a|s|^2 + 8a|s| - 4a & \text{if } 1 < |s| \leq 2 \\
0 & \text{if } |s| > 2
\end{cases}
\end{equation}

For bicubic interpolation, a $4 \times 4$ neighborhood is sampled around each coordinate, while tricubic interpolation samples a $4 \times 4 \times 4$ neighborhood. The final interpolated value is computed as the separable product of 1D kernel evaluations along each dimension.

All interpolation operations support oversampling, where coordinates are scaled by a factor $> 1$ to sample from reconstructions that were previously zero-padded in real space to increase the Fourier space sampling rate. This improves interpolation accuracy at constant computational load, but at the expense of increased memory usage. The sampling grid is sparse, matching the box size of the unpadded reconstruction. Because of this, no additional real-space cropping is required for the final projections.

A hard low-pass filter can be applied during the projection operation.

Shifts can be applied as part of the forward and backward projection operations. This is achieved by applying a phase modulation to the complex Fourier space components according to the Fourier shift theorem: $F\{f(\mathbf{x} - \mathbf{s})\} = F\{f(\mathbf{x})\} \cdot e^{-i2\pi \mathbf{k} \cdot \mathbf{s}}$, where $\mathbf{k}$ are the Fourier coordinates and $\mathbf{s}$ is the shift vector.

\subsection{Friedel Symmetry}
The Fourier transform of a real-valued function exhibits Friedel symmetry, characterized by the relationship:

\begin{equation}
\label{eq:friedel_symmetry}
F(-\mathbf{k}) = F^*(\mathbf{k})
\end{equation}

where $F^*(\mathbf{k})$ denotes the complex conjugate at frequency $\mathbf{k}$. This symmetry property allows PyTorch's RFFT format to store only the non-negative frequency components along the last dimension, yielding the characteristic $[\ldots, N/2+1]$ shape.

However, the zero-frequency axis (where $k_c = 0$) requires careful handling during projection operations to maintain mathematical consistency and prevent double-counting:

\textbf{Forward Pass}: Standard Friedel symmetry lookup is applied when sampling negative $k_c$ coordinates. For $k_c < 0$, the implementation accesses $F(-k_c, -k_r)$ and returns its complex conjugate $F^*(-k_c, -k_r)$ according to equation~\ref{eq:friedel_symmetry}.

\textbf{Backward Pass}: To avoid over-representation in gradient accumulation, the implementation skips processing the negative half of the zero-frequency axis (specifically, components where $k_c = 0$ and $k_r < 0$), since these are present twice in the RFFT-formatted gradient.

\textbf{Accumulation on Zero-Frequency Axis}: When projection operations result in values that must be accumulated at positions on the zero-frequency axis, the implementation maintains Friedel symmetry by inserting contributions at both the target location $(k_r, 0)$ and its symmetric counterpart $(-k_r, 0)$.

\subsection{Implementation Architecture}

\texttt{torch-projectors} follows PyTorch's hybrid C++/Python extension architecture using the \texttt{TORCH\_LIBRARY} registration system. The library implements a multi-backend design with separate optimized kernels for CPU, Apple Silicon (MPS), and CUDA devices.

Each backend maintains its own implementation in dedicated directories (\texttt{csrc/cpu/}, \texttt{csrc/mps/}, \texttt{csrc/cuda/}), with common utilities factored into \texttt{csrc/cpu/common/}. All projection operators are registered in a unified \texttt{torch\_projectors} namespace using \texttt{TORCH\_LIBRARY} declarations, enabling automatic device dispatch through PyTorch's operator registration system.

The Python interface wraps C++ operators with custom \texttt{torch.autograd.Function} classes that handle gradient computation. Library initialization loads the compiled C++ extension and registers Python autograd functions using \texttt{torch.library.register\_autograd}. This architecture provides seamless integration with PyTorch's automatic differentiation while maintaining high performance through backend-specific optimizations.

\subsection{2D $\rightarrow$ 2D Forward Projection}

The 2D forward projection operator generates rotated 2D projections from 2D Fourier-space reconstructions. Given a reconstruction tensor $[B, N, N/2+1]$ and rotation matrices $[B_{rot}, P, 2, 2]$, the operator produces projections $[B, P, N_{out}, N_{out}/2+1]$ by sampling the reconstruction at transformed Fourier coordinates.

\subsubsection{Forward Pass}
For each output Fourier coordinate $\mathbf{k}_{out}$ in the projection, the operator computes the corresponding coordinate in the reconstruction as:

\begin{equation}
\label{eq:coord_transform}
\mathbf{k}_{rec} = \mathbf{R}^{-1} \mathbf{k}_{out}
\end{equation}

where $\mathbf{R}$ is the 2D rotation matrix. The projection value is obtained by interpolating the reconstruction at $\mathbf{k}_{rec}$, then applying phase modulation for translation:

\begin{equation}
\label{eq:forward_projection}
P(\mathbf{k}_{out}) = F_{rec}(\mathbf{k}_{rec}) \cdot e^{-i 2\pi \mathbf{k}_{out} \cdot \mathbf{s}}
\end{equation}

where $\mathbf{s}$ is the shift vector.

\subsubsection{Backward Pass}
During backpropagation, the incoming projection gradients are first phase-modulated using the conjugate shift:

\begin{equation}
\label{eq:conjugate_phase}
\nabla P' = \nabla P \cdot e^{i 2\pi \mathbf{k}_{out} \cdot \mathbf{s}}
\end{equation}

Reconstruction gradients are then accumulated via transposed interpolation operations at the transformed coordinates $\mathbf{k}_{rec}$.

Analytical rotation matrix gradients require careful application of the chain rule: Since $\mathbf{k}_{rec} = \mathbf{R}^{-1} \mathbf{k}_{out}$, the gradient with respect to rotation matrix element $R_{ij}$ is:

\begin{equation}
\label{eq:rotation_gradient_chain}
\frac{\partial P}{\partial R_{ij}} = \frac{\partial F_{rec}(\mathbf{k}_{rec})}{\partial \mathbf{k}_{rec}} \cdot \frac{\partial \mathbf{k}_{rec}}{\partial R_{ij}}
\end{equation}

For bilinear interpolation, spatial derivatives are computed from the 2×2 sample grid using separable linear kernel derivatives:

\begin{align}
\frac{\partial F}{\partial r} &= \sum_{i=0}^{1} \sum_{j=0}^{1} p_{ij} \cdot l'(r_f-i) \cdot l(c_f-j) \label{eq:bilinear_grad_r} \\
\frac{\partial F}{\partial c} &= \sum_{i=0}^{1} \sum_{j=0}^{1} p_{ij} \cdot l(r_f-i) \cdot l'(c_f-j) \label{eq:bilinear_grad_c}
\end{align}

where $l(s) = 1-|s|$ for $|s| \leq 1$ and $l'(s) = \text{sign}(s)$ for $|s| < 1$.

For bicubic interpolation, derivatives follow from separable Catmull-Rom kernel derivatives over the 4×4 neighborhood:

\begin{align}
\label{eq:bicubic_gradients}
\frac{\partial F}{\partial r} &= \sum_{i=-1}^{2} \sum_{j=-1}^{2} p_{ij} \cdot w'(r_f-i) \cdot w(c_f-j) \\
\frac{\partial F}{\partial c} &= \sum_{i=-1}^{2} \sum_{j=-1}^{2} p_{ij} \cdot w(r_f-i) \cdot w'(c_f-j)
\end{align}

where $w'(s)$ is the cubic kernel derivative. The coordinate transformation derivatives $\frac{\partial \mathbf{k}_{rec}}{\partial R_{ij}}$ are computed from the matrix inverse relationships.

Shift gradients follow from the phase modulation derivative:

\begin{equation}
\label{eq:shift_gradient}
\frac{\partial P}{\partial \mathbf{s}} = -i 2\pi \mathbf{k}_{out} P(\mathbf{k}_{out})
\end{equation}

\subsection{2D $\rightarrow$ 2D Backward Projection}

The 2D backward projection operator is the mathematical adjoint of the forward projection operation. It accumulates 2D Fourier-space projections $[B, P, N, N/2+1]$ into 2D reconstructions $[B, N_{rec}, N_{rec}/2+1]$, where the reconstruction size accounts for oversampling: $N_{rec} = N \cdot \text{oversampling}$.

\subsubsection{Forward Pass}
For each projection coordinate $\mathbf{k}_{proj} = (k_r, k_c)$, the corresponding coordinate in the 2D reconstruction is computed using equation~\ref{eq:coord_transform} with the same 2×2 rotation matrix $\mathbf{R}$. The projection value is then accumulated into the reconstruction at the transformed coordinate. When shifts are present, the projection data is first conjugate phase-modulated:

\begin{equation}
\label{eq:backproj_conjugate_phase}
P'(\mathbf{k}_{proj}) = P(\mathbf{k}_{proj}) \cdot e^{i 2\pi \mathbf{k}_{proj} \cdot \mathbf{s}}
\end{equation}

where the conjugate phase ensures the mathematical adjoint relationship with forward projection.

The accumulation process uses 2D interpolation kernels in transpose mode. For bilinear interpolation, contributions are distributed to the 2×2 neighborhood around each fractional coordinate. For bicubic interpolation, contributions are spread across the 4×4 neighborhood according to the Catmull-Rom weights defined in equations~\ref{eq:bilinear_grad_r}--\ref{eq:bilinear_grad_c} and ~\ref{eq:bicubic_gradients}.

Optional weight accumulation supports applications requiring per-Fourier component weights, such as CTF correction in cryo-EM. Weights are accumulated using the absolute values of the 2D interpolation kernel weights, maintaining a reference for downstream normalization (e.g. for Wiener-like filters).

\subsubsection{Backward Pass}
During backpropagation, projection gradients are computed using forward projection of reconstruction gradients, exploiting the adjoint relationship. Rotation matrix gradients follow equation~\ref{eq:rotation_gradient_chain} with spatial derivatives computed using equations~\ref{eq:bilinear_grad_r}--\ref{eq:bilinear_grad_c} for bilinear interpolation or equation~\ref{eq:bicubic_gradients} for bicubic interpolation. Shift gradients are computed using equation~\ref{eq:shift_gradient}, applied to the accumulated reconstruction data.

\subsection{3D $\rightarrow$ 2D Forward Projection}

The 3D→2D forward projection operator implements the Central Slice Theorem, generating 2D projections from 3D Fourier-space reconstructions. Given a 3D reconstruction tensor $[B, D, H, W/2+1]$, 3×3 rotation matrices $[B, P, 3, 3]$, and optional 2D shifts $[B, P, 2]$, the operator produces projections $[B, P, H_{out}, W_{out}/2+1]$ by sampling central slices through the 3D volume.

\subsubsection{Forward Pass}
For each output projection coordinate $\mathbf{k}_{proj} = (k_r, k_c)$, the operator extends it to 3D by setting the third coordinate to zero: $\mathbf{k}_{3D} = (k_c, k_r, 0)$. This implements the central slice through the origin required by the Central Slice Theorem. 

The corresponding coordinate in the 3D reconstruction is computed using the 3×3 rotation matrix:

\begin{equation}
\label{eq:coord_transform_3d}
\mathbf{k}_{rec} = \mathbf{R}^{-1} \mathbf{k}_{3D}
\end{equation}

where $\mathbf{R}$ is the 3×3 rotation matrix. The projection value is obtained by trilinear or tricubic interpolation in the 3D volume at $\mathbf{k}_{rec}$, then applying phase modulation for translation:

\begin{equation}
\label{eq:forward_projection_3d}
P(\mathbf{k}_{proj}) = F_{rec}(\mathbf{k}_{rec}) \cdot e^{-i 2\pi \mathbf{k}_{proj} \cdot \mathbf{s}}
\end{equation}

where $\mathbf{s}$ is the 2D shift vector applied to the projection coordinates.

\subsubsection{Backward Pass}
During backpropagation, the incoming projection gradients are first phase-modulated using the conjugate shift applied to the 2D projection coordinates (using equation~\ref{eq:conjugate_phase}).

Reconstruction gradients are then accumulated via 3D transposed interpolation operations at the transformed coordinates $\mathbf{k}_{rec}$.

Analytical 3×3 rotation matrix gradients require extension of the chain rule to three dimensions: Since $\mathbf{k}_{rec} = \mathbf{R}^{-1} \mathbf{k}_{3D}$, the gradient with respect to rotation matrix element $R_{ij}$ is:

\begin{equation}
\label{eq:rotation_gradient_chain_3d}
\frac{\partial P}{\partial R_{ij}} = \frac{\partial F_{rec}(\mathbf{k}_{rec})}{{\partial \mathbf{k}_{rec}}} \cdot \frac{\partial \mathbf{k}_{rec}}{\partial R_{ij}}
\end{equation}

For trilinear interpolation, spatial derivatives are computed from the 2×2×2 sample grid using separable linear kernel derivatives:

\begin{align}
\frac{\partial F}{\partial d} &= \sum_{i=0}^{1} \sum_{j=0}^{1} \sum_{k=0}^{1} p_{ijk} \cdot l'(d_f-i) \cdot l(r_f-j) \cdot l(c_f-k) \label{eq:trilinear_grad_d} \\
\frac{\partial F}{\partial r} &= \sum_{i=0}^{1} \sum_{j=0}^{1} \sum_{k=0}^{1} p_{ijk} \cdot l(d_f-i) \cdot l'(r_f-j) \cdot l(c_f-k) \label{eq:trilinear_grad_r} \\
\frac{\partial F}{\partial c} &= \sum_{i=0}^{1} \sum_{j=0}^{1} \sum_{k=0}^{1} p_{ijk} \cdot l(d_f-i) \cdot l(r_f-j) \cdot l'(c_f-k) \label{eq:trilinear_grad_c}
\end{align}

where $l(s) = 1-|s|$ for $|s| \leq 1$ and $l'(s) = \text{sign}(s)$ for $|s| < 1$.

For tricubic interpolation, derivatives follow from separable Catmull-Rom kernel derivatives over the 4×4×4 neighborhood:

\begin{align}
\label{eq:tricubic_gradients}
\frac{\partial F}{\partial d} &= \sum_{i=-1}^{2} \sum_{j=-1}^{2} \sum_{k=-1}^{2} p_{ijk} \cdot w'(d_f-i) \cdot w(r_f-j) \cdot w(c_f-k) \\
\frac{\partial F}{\partial r} &= \sum_{i=-1}^{2} \sum_{j=-1}^{2} \sum_{k=-1}^{2} p_{ijk} \cdot w(d_f-i) \cdot w'(r_f-j) \cdot w(c_f-k) \\
\frac{\partial F}{\partial c} &= \sum_{i=-1}^{2} \sum_{j=-1}^{2} \sum_{k=-1}^{2} p_{ijk} \cdot w(d_f-i) \cdot w(r_f-j) \cdot w'(c_f-k)
\end{align}

where $w'(s)$ is the cubic kernel derivative. The coordinate transformation derivatives $\frac{\partial \mathbf{k}_{rec}}{\partial R_{ij}}$ are computed from the matrix inverse relationships.

Shift gradients follow from the phase modulation derivative applied to the 2D projection coordinates:

\begin{equation}
\label{eq:shift_gradient_3d}
\frac{\partial P}{\partial \mathbf{s}} = -i 2\pi \mathbf{k}_{proj} P(\mathbf{k}_{proj})
\end{equation}

\subsection{2D $\rightarrow$ 3D Backward Projection}

The 2D→3D backward projection operator is the mathematical adjoint of the 3D→2D forward projection operation. It accumulates 2D Fourier-space projections $[B, P, H, W/2+1]$ into 3D reconstructions $[B, D, H_{rec}, W_{rec}/2+1]$, where the reconstruction dimensions account for oversampling and form a cubic volume: $D = H_{rec} = W_{rec} = H \cdot \text{oversampling}$.

\subsubsection{Forward Pass}
For each projection coordinate $\mathbf{k}_{proj} = (k_r, k_c)$, the operator extends it to 3D by setting the third coordinate to zero, implementing the central slice through the origin required by the Central Slice Theorem:

\begin{equation}
\label{eq:coord_extension_3d}
\mathbf{k}_{3D} = (k_c, k_r, 0)
\end{equation}

The 3D sampling coordinate in the reconstruction is computed using equation~\ref{eq:coord_transform_3d} with the same 3×3 rotation matrix $\mathbf{R}$. The projection value is then accumulated into the reconstruction at the transformed coordinate. When shifts are present, the projection data is first conjugate phase-modulated using equation~\ref{eq:backproj_conjugate_phase} to ensure the mathematical adjoint relationship.

The accumulation process uses 3D interpolation kernels in transpose mode. For trilinear interpolation, contributions are distributed to the 2×2×2 neighborhood around each fractional coordinate. For tricubic interpolation, contributions are spread across the 4×4×4 neighborhood according to the Catmull-Rom weights defined in equations~\ref{eq:trilinear_grad_d}--\ref{eq:trilinear_grad_c} and ~\ref{eq:tricubic_gradients}.

Optional weight accumulation supports applications requiring per-Fourier component weights, such as CTF correction in cryo-EM. Weights are accumulated using the absolute values of the 3D interpolation kernel weights, with proper handling of 3D Friedel symmetry for real-valued reconstructions.

\subsubsection{Backward Pass}
During backpropagation, projection gradients are computed using 3D→2D forward projection of reconstruction gradients, exploiting the adjoint relationship. The 3×3 rotation matrix gradients follow equation~\ref{eq:rotation_gradient_chain_3d} with spatial derivatives computed using equations~\ref{eq:trilinear_grad_d}--\ref{eq:trilinear_grad_c} for trilinear interpolation or equation~\ref{eq:tricubic_gradients} for tricubic interpolation. Shift gradients are computed using equation~\ref{eq:shift_gradient_3d}, applied to the 2D projection coordinates since shifts affect only the 2D projection plane.

\section{Results}
Performance benchmarks were conducted on three representative platforms: NVIDIA H100 CUDA, Apple M4 CPU, and Apple M4 MPS. Throughput was measured in thousands of projections per second across systematic parameter sweeps including box sizes (32, 128, 256 pixels), batch dimensions (1, 8), pose counts (8, 128, 2048), and interpolation methods (linear, cubic). Each test case used three warm-up runs followed by 15 timed runs, with median throughput reported.

\subsection{Performance Benchmarks}

Because computation is parallelized per-projection (1 thread per projection on CPU, 1 thread block per projection on MPS and CUDA), projection throughput exhibits strong scaling with batch size and pose count across all platforms (Tables~\ref{tab:project-2d-performance}--\ref{tab:backproject-2d-to-3d-performance}). The platforms show significant performance differences, with CUDA implementations achieving the highest throughput, followed by MPS and CPU. The difference between CPU and CUDA is over 2 orders of magnitude, owing to the much higher computational throughput and memory bandwidth of the H100 chip.

Cubic interpolation requires $4\times$ more samples than linear interpolation in 2D, and $8\times$ more in 3D. In practice, the penalty is smaller than expected because sample groups in the fastest dimension are co-located in memory, reducing the cost difference between reading 2 and 4 samples. Interestingly, despite a very scattered memory access pattern, the CUDA backend exceeds the H100 chip's theoretical bandwidth: $3.4\,\mathrm{TB/s}$ corresponds to $3{.}4\times 10^6$ projections/second with 4 fp32 reads per pixel, yet $3{.}7\times 10^6$ are achieved in the most favorable case here. This suggests the reference data fit entirely into the chip's cache.

The backward pass in forward projection and the forward pass in backward projection are significantly slower than their counterparts because they involve scatter operations, requiring costly atomic writes to global memory.

\begin{table}[htbp]
\centering
\caption{2D $\rightarrow$ 2D forward projection throughput in $10^3$ projections/second}
\label{tab:project-2d-performance}
\begin{tabular}{r@{\hspace{3mm}} c@{\hspace{2mm}} r@{\hspace{3mm}} c@{\hspace{2mm}} | *{6}{>{\raggedleft\arraybackslash}p{1.5cm}}}
\toprule
\multirow{2}{*}[-0.6em]{Box} & \multirow{2}{*}[-0.6em]{Batch} & \multirow{2}{*}[-0.6em]{Poses} & \multirow{2}{*}[-0.6em]{Interp.} & \multicolumn{2}{c}{H100: CUDA} & \multicolumn{2}{c}{M4: CPU} & \multicolumn{2}{c}{M4: MPS} \\
& & & & Fwd & Fwd+Bwd & Fwd & Fwd+Bwd & Fwd & Fwd+Bwd \\
\midrule
\multirow{12}{*}[-1.2em]{32} & \multirow{6}{*}[-0.6em]{1} & \multirow{2}{*}[-0.1em]{8} & Linear & 201.7 & 35.7 & 119.9 & 39.3 & 16.3 & 9.0 \\
 &  &  & Cubic & 198.4 & 37.4 & 82.3 & 22.3 & 32.7 & 7.2 \\
\cmidrule{3-10}
 &  & \multirow{2}{*}[-0.1em]{128} & Linear & 3248.5 & 580.4 & 629.2 & 71.2 & 462.9 & 81.5 \\
 &  &  & Cubic & 3208.1 & 393.9 & 392.3 & 37.2 & 397.6 & 70.5 \\
\cmidrule{3-10}
 &  & \multirow{2}{*}[-0.1em]{2048} & Linear & 44496.6 & 3223.5 & 1296.7 & 85.0 & 2910.5 & 624.0 \\
 &  &  & Cubic & 32044.1 & 1026.4 & 520.9 & 40.2 & 1478.3 & 184.8 \\
\cmidrule{3-10}
\cmidrule{2-10}
 & \multirow{6}{*}[-0.6em]{8} & \multirow{2}{*}[-0.1em]{8} & Linear & 1541.8 & 303.1 & 414.8 & 107.7 & 258.8 & 67.5 \\
 &  &  & Cubic & 1559.5 & 289.1 & 316.4 & 70.5 & 240.2 & 43.9 \\
\cmidrule{3-10}
 &  & \multirow{2}{*}[-0.1em]{128} & Linear & 25379.8 & 3572.7 & 1100.1 & 225.9 & 1999.5 & 623.4 \\
 &  &  & Cubic & 19847.7 & 1738.9 & 480.2 & 107.0 & 1174.9 & 373.2 \\
\cmidrule{3-10}
 &  & \multirow{2}{*}[-0.1em]{2048} & Linear & 134947.7 & 4216.0 & 1233.7 & 264.2 & 8660.8 & 1275.2 \\
 &  &  & Cubic & 60536.6 & 1130.9 & 494.6 & 116.5 & 5542.4 & 555.3 \\
\midrule
\multirow{12}{*}[-1.2em]{128} & \multirow{6}{*}[-0.6em]{1} & \multirow{2}{*}[-0.1em]{8} & Linear & 152.7 & 36.2 & 41.0 & 7.2 & 23.2 & 4.7 \\
 &  &  & Cubic & 106.1 & 22.0 & 24.2 & 3.8 & 19.6 & 4.5 \\
\cmidrule{3-10}
 &  & \multirow{2}{*}[-0.1em]{128} & Linear & 2396.0 & 412.4 & 103.9 & 12.3 & 189.7 & 63.3 \\
 &  &  & Cubic & 1576.4 & 169.8 & 36.1 & 5.4 & 93.4 & 25.6 \\
\cmidrule{3-10}
 &  & \multirow{2}{*}[-0.1em]{2048} & Linear & 11067.6 & 776.7 & 94.6 & 12.7 & 524.3 & 80.2 \\
 &  &  & Cubic & 4302.8 & 231.1 & 34.4 & 5.4 & 349.8 & 32.9 \\
\cmidrule{3-10}
\cmidrule{2-10}
 & \multirow{6}{*}[-0.6em]{8} & \multirow{2}{*}[-0.1em]{8} & Linear & 1215.1 & 254.4 & 84.0 & 14.8 & 129.3 & 47.5 \\
 &  &  & Cubic & 794.1 & 139.8 & 33.5 & 6.9 & 71.9 & 21.7 \\
\cmidrule{3-10}
 &  & \multirow{2}{*}[-0.1em]{128} & Linear & 8921.7 & 872.0 & 97.2 & 18.4 & 571.3 & 84.0 \\
 &  &  & Cubic & 3892.2 & 259.6 & 34.4 & 7.7 & 358.8 & 35.4 \\
\cmidrule{3-10}
 &  & \multirow{2}{*}[-0.1em]{2048} & Linear & 14040.0 & 824.9 & 92.0 & 16.6 & 501.7 & 86.1 \\
 &  &  & Cubic & 4844.0 & 239.3 & 34.5 & 7.1 & 335.7 & 36.6 \\
\midrule
\multirow{12}{*}[-1.2em]{256} & \multirow{6}{*}[-0.6em]{1} & \multirow{2}{*}[-0.1em]{8} & Linear & 81.4 & 21.2 & 19.2 & 2.1 & 14.3 & 5.0 \\
 &  &  & Cubic & 40.1 & 8.7 & 7.2 & 1.0 & 8.3 & 2.5 \\
\cmidrule{3-10}
 &  & \multirow{2}{*}[-0.1em]{128} & Linear & 1212.3 & 204.2 & 28.7 & 3.2 & 69.6 & 20.3 \\
 &  &  & Cubic & 599.2 & 67.0 & 9.2 & 1.5 & 91.2 & 8.0 \\
\cmidrule{3-10}
 &  & \multirow{2}{*}[-0.1em]{2048} & Linear & 3333.2 & 269.2 & 23.6 & 3.3 & 125.6 & 21.8 \\
 &  &  & Cubic & 1208.7 & 80.7 & 8.8 & 1.5 & 86.9 & 9.2 \\
\cmidrule{3-10}
\cmidrule{2-10}
 & \multirow{6}{*}[-0.6em]{8} & \multirow{2}{*}[-0.1em]{8} & Linear & 630.9 & 145.4 & 28.2 & 4.8 & 60.9 & 19.4 \\
 &  &  & Cubic & 306.3 & 54.3 & 9.2 & 1.9 & 57.3 & 8.1 \\
\cmidrule{3-10}
 &  & \multirow{2}{*}[-0.1em]{128} & Linear & 2999.6 & 275.1 & 24.7 & 4.7 & 131.2 & 21.6 \\
 &  &  & Cubic & 1128.9 & 86.1 & 8.9 & 1.9 & 88.9 & 9.1 \\
\cmidrule{3-10}
 &  & \multirow{2}{*}[-0.1em]{2048} & Linear & 3695.1 & 266.8 & 23.1 & 4.1 & 126.4 & 22.3 \\
 &  &  & Cubic & 1292.7 & 83.0 & 8.7 & 1.8 & 87.9 & 9.1 \\
\bottomrule
\end{tabular}
\end{table}

\begin{table}[htbp]
\centering
\caption{2D $\rightarrow$ 2D backward projection throughput in $10^3$ projections/second}
\label{tab:backproject-2d-performance}
\begin{tabular}{r@{\hspace{3mm}} c@{\hspace{2mm}} r@{\hspace{3mm}} c@{\hspace{2mm}} | *{6}{>{\raggedleft\arraybackslash}p{1.5cm}}}
\toprule
\multirow{2}{*}[-0.6em]{Box} & \multirow{2}{*}[-0.6em]{Batch} & \multirow{2}{*}[-0.6em]{Poses} & \multirow{2}{*}[-0.6em]{Interp.} & \multicolumn{2}{c}{H100: CUDA} & \multicolumn{2}{c}{M4: CPU} & \multicolumn{2}{c}{M4: MPS} \\
& & & & Fwd & Fwd+Bwd & Fwd & Fwd+Bwd & Fwd & Fwd+Bwd \\
\midrule
\multirow{12}{*}[-1.2em]{32} & \multirow{6}{*}[-0.6em]{1} & \multirow{2}{*}[-0.1em]{8} & Linear & 67.5 & 21.4 & 37.4 & 26.0 & 20.0 & 7.5 \\
 &  &  & Cubic & 97.9 & 25.9 & 17.6 & 14.2 & 10.3 & 5.5 \\
\cmidrule{3-10}
 &  & \multirow{2}{*}[-0.1em]{128} & Linear & 1284.4 & 389.5 & 56.0 & 50.0 & 101.0 & 65.0 \\
 &  &  & Cubic & 557.7 & 278.1 & 27.8 & 25.3 & 117.2 & 78.3 \\
\cmidrule{3-10}
 &  & \multirow{2}{*}[-0.1em]{2048} & Linear & 3395.1 & 2420.4 & 62.8 & 58.1 & 619.8 & 516.5 \\
 &  &  & Cubic & 940.2 & 845.5 & 28.7 & 27.0 & 159.5 & 150.1 \\
\cmidrule{3-10}
\cmidrule{2-10}
 & \multirow{6}{*}[-0.6em]{8} & \multirow{2}{*}[-0.1em]{8} & Linear & 783.4 & 202.8 & 197.9 & 112.6 & 137.4 & 52.7 \\
 &  &  & Cubic & 651.6 & 195.0 & 105.0 & 69.7 & 153.3 & 65.9 \\
\cmidrule{3-10}
 &  & \multirow{2}{*}[-0.1em]{128} & Linear & 5799.6 & 2553.1 & 249.0 & 185.8 & 1189.9 & 684.3 \\
 &  &  & Cubic & 2001.5 & 1364.2 & 117.4 & 91.5 & 461.4 & 346.0 \\
\cmidrule{3-10}
 &  & \multirow{2}{*}[-0.1em]{2048} & Linear & 3772.6 & 3487.9 & 255.3 & 198.5 & 2130.1 & 1551.2 \\
 &  &  & Cubic & 961.2 & 933.8 & 114.4 & 90.8 & 561.5 & 497.3 \\
\midrule
\multirow{12}{*}[-1.2em]{128} & \multirow{6}{*}[-0.6em]{1} & \multirow{2}{*}[-0.1em]{8} & Linear & 67.1 & 23.0 & 8.2 & 6.1 & 14.0 & 7.4 \\
 &  &  & Cubic & 29.5 & 15.5 & 3.8 & 3.1 & 5.9 & 4.4 \\
\cmidrule{3-10}
 &  & \multirow{2}{*}[-0.1em]{128} & Linear & 550.7 & 270.7 & 8.6 & 7.5 & 84.1 & 59.9 \\
 &  &  & Cubic & 169.8 & 127.8 & 4.0 & 3.5 & 24.7 & 21.6 \\
\cmidrule{3-10}
 &  & \multirow{2}{*}[-0.1em]{2048} & Linear & 777.0 & 664.3 & 8.8 & 7.8 & 120.6 & 92.9 \\
 &  &  & Cubic & 200.7 & 186.4 & 3.5 & 3.1 & 30.7 & 27.8 \\
\cmidrule{3-10}
\cmidrule{2-10}
 & \multirow{6}{*}[-0.6em]{8} & \multirow{2}{*}[-0.1em]{8} & Linear & 469.8 & 175.2 & 16.1 & 11.2 & 64.6 & 39.5 \\
 &  &  & Cubic & 175.2 & 104.8 & 7.4 & 5.6 & 24.1 & 19.3 \\
\cmidrule{3-10}
 &  & \multirow{2}{*}[-0.1em]{128} & Linear & 955.9 & 729.1 & 17.0 & 13.3 & 126.5 & 94.8 \\
 &  &  & Cubic & 251.2 & 223.2 & 7.1 & 5.8 & 32.8 & 29.4 \\
\cmidrule{3-10}
 &  & \multirow{2}{*}[-0.1em]{2048} & Linear & 837.0 & 749.2 & 15.7 & 12.3 & 135.3 & 101.8 \\
 &  &  & Cubic & 210.9 & 198.8 & 6.6 & 5.4 & 34.3 & 30.7 \\
\midrule
\multirow{12}{*}[-1.2em]{256} & \multirow{6}{*}[-0.6em]{1} & \multirow{2}{*}[-0.1em]{8} & Linear & 27.0 & 13.4 & 1.9 & 1.5 & 4.7 & 3.4 \\
 &  &  & Cubic & 8.6 & 6.0 & 1.1 & 0.9 & 2.0 & 1.7 \\
\cmidrule{3-10}
 &  & \multirow{2}{*}[-0.1em]{128} & Linear & 227.0 & 147.7 & 2.7 & 2.3 & 26.6 & 20.5 \\
 &  &  & Cubic & 62.7 & 52.0 & 1.3 & 1.1 & 7.0 & 6.4 \\
\cmidrule{3-10}
 &  & \multirow{2}{*}[-0.1em]{2048} & Linear & 294.8 & 250.2 & 2.5 & 2.1 & 33.5 & 25.3 \\
 &  &  & Cubic & 75.1 & 68.9 & 1.1 & 1.0 & 8.5 & 7.7 \\
\cmidrule{3-10}
\cmidrule{2-10}
 & \multirow{6}{*}[-0.6em]{8} & \multirow{2}{*}[-0.1em]{8} & Linear & 193.9 & 101.6 & 4.3 & 3.3 & 24.2 & 18.3 \\
 &  &  & Cubic & 55.0 & 40.6 & 1.9 & 1.5 & 6.7 & 6.0 \\
\cmidrule{3-10}
 &  & \multirow{2}{*}[-0.1em]{128} & Linear & 305.4 & 249.5 & 4.2 & 3.3 & 33.1 & 25.1 \\
 &  &  & Cubic & 78.6 & 71.2 & 1.8 & 1.4 & 8.4 & 7.6 \\
\cmidrule{3-10}
 &  & \multirow{2}{*}[-0.1em]{2048} & Linear & 297.3 & 258.1 & 4.1 & 3.2 & 34.7 & 25.9 \\
 &  &  & Cubic & 74.8 & 69.3 & 1.8 & 1.4 & 8.8 & 7.7 \\
\bottomrule
\end{tabular}
\end{table}

\begin{table}[htbp]
\centering
\caption{3D $\rightarrow$ 2D forward projection throughput in $10^3$ projections/second}
\label{tab:project-3d-to-2d-performance}
\begin{tabular}{r@{\hspace{3mm}} c@{\hspace{2mm}} r@{\hspace{3mm}} c@{\hspace{2mm}} | *{6}{>{\raggedleft\arraybackslash}p{1.5cm}}}
\toprule
\multirow{2}{*}[-0.6em]{Box} & \multirow{2}{*}[-0.6em]{Batch} & \multirow{2}{*}[-0.6em]{Poses} & \multirow{2}{*}[-0.6em]{Interp.} & \multicolumn{2}{c}{H100: CUDA} & \multicolumn{2}{c}{M4: CPU} & \multicolumn{2}{c}{M4: MPS} \\
& & & & Fwd & Fwd+Bwd & Fwd & Fwd+Bwd & Fwd & Fwd+Bwd \\
\midrule
\multirow{12}{*}[-1.2em]{32} & \multirow{6}{*}[-0.6em]{1} & \multirow{2}{*}[-0.1em]{8} & Linear & 37.2 & 16.2 & 117.6 & 30.4 & 40.9 & 10.2 \\
 &  &  & Cubic & 31.8 & 15.6 & 69.1 & 15.4 & 31.9 & 5.0 \\
\cmidrule{3-10}
 &  & \multirow{2}{*}[-0.1em]{128} & Linear & 576.8 & 261.4 & 524.3 & 77.4 & 569.2 & 160.6 \\
 &  &  & Cubic & 516.8 & 205.0 & 199.7 & 20.4 & 487.1 & 76.9 \\
\cmidrule{3-10}
 &  & \multirow{2}{*}[-0.1em]{2048} & Linear & 8579.5 & 2536.7 & 987.2 & 98.1 & 5788.0 & 748.6 \\
 &  &  & Cubic & 4821.2 & 637.3 & 229.5 & 22.4 & 1906.2 & 146.1 \\
\cmidrule{3-10}
\cmidrule{2-10}
 & \multirow{6}{*}[-0.6em]{8} & \multirow{2}{*}[-0.1em]{8} & Linear & 1596.3 & 307.7 & 352.1 & 81.5 & 286.9 & 82.3 \\
 &  &  & Cubic & 1071.7 & 250.0 & 172.3 & 32.0 & 276.5 & 55.5 \\
\cmidrule{3-10}
 &  & \multirow{2}{*}[-0.1em]{128} & Linear & 22239.1 & 3546.4 & 843.9 & 159.5 & 3735.5 & 680.9 \\
 &  &  & Cubic & 9301.1 & 1009.3 & 222.6 & 40.3 & 1578.9 & 153.3 \\
\cmidrule{3-10}
 &  & \multirow{2}{*}[-0.1em]{2048} & Linear & 86681.4 & 6816.6 & 937.3 & 188.1 & 7367.9 & 877.6 \\
 &  &  & Cubic & 17133.8 & 993.2 & 226.9 & 41.6 & 2045.2 & 166.1 \\
\midrule
\multirow{12}{*}[-1.2em]{128} & \multirow{6}{*}[-0.6em]{1} & \multirow{2}{*}[-0.1em]{8} & Linear & 24.1 & 12.2 & 38.4 & 5.9 & 31.1 & 7.5 \\
 &  &  & Cubic & 11.6 & 5.5 & 12.2 & 1.9 & 21.8 & 2.5 \\
\cmidrule{3-10}
 &  & \multirow{2}{*}[-0.1em]{128} & Linear & 393.8 & 176.9 & 70.8 & 8.9 & 336.7 & 47.4 \\
 &  &  & Cubic & 178.3 & 56.6 & 14.7 & 2.3 & 109.9 & 8.6 \\
\cmidrule{3-10}
 &  & \multirow{2}{*}[-0.1em]{2048} & Linear & 3038.7 & 556.9 & 67.7 & 9.1 & 392.9 & 52.9 \\
 &  &  & Cubic & 632.1 & 88.5 & 14.6 & 2.2 & 126.3 & 9.7 \\
\cmidrule{3-10}
\cmidrule{2-10}
 & \multirow{6}{*}[-0.6em]{8} & \multirow{2}{*}[-0.1em]{8} & Linear & 916.8 & 190.1 & 49.5 & 6.7 & 196.0 & 20.0 \\
 &  &  & Cubic & 167.4 & 46.2 & 13.8 & 2.2 & 92.8 & 6.5 \\
\cmidrule{3-10}
 &  & \multirow{2}{*}[-0.1em]{128} & Linear & 5256.8 & 579.1 & 55.7 & 9.2 & 313.9 & 47.3 \\
 &  &  & Cubic & 917.1 & 87.6 & 14.0 & 2.3 & 126.6 & 9.5 \\
\cmidrule{3-10}
 &  & \multirow{2}{*}[-0.1em]{2048} & Linear & 7729.0 & 645.8 & 57.3 & 9.1 & 304.1 & 51.6 \\
 &  &  & Cubic & 1110.4 & 95.2 & 12.8 & 2.1 & 126.6 & 9.7 \\
\midrule
\multirow{12}{*}[-1.2em]{256} & \multirow{6}{*}[-0.6em]{1} & \multirow{2}{*}[-0.1em]{8} & Linear & 14.7 & 7.5 & 11.3 & 1.0 & 24.3 & 2.3 \\
 &  &  & Cubic & 3.7 & 1.8 & 2.9 & 0.4 & 8.0 & 0.7 \\
\cmidrule{3-10}
 &  & \multirow{2}{*}[-0.1em]{128} & Linear & 231.2 & 89.0 & 11.1 & 1.8 & 70.1 & 10.6 \\
 &  &  & Cubic & 57.7 & 17.6 & 2.9 & 0.5 & 30.4 & 2.2 \\
\cmidrule{3-10}
 &  & \multirow{2}{*}[-0.1em]{2048} & Linear & 929.5 & 149.1 & 11.0 & 1.9 & 62.2 & 12.4 \\
 &  &  & Cubic & 164.6 & 23.0 & 3.2 & 0.5 & 30.2 & 2.4 \\
\cmidrule{3-10}
\cmidrule{2-10}
 & \multirow{6}{*}[-0.6em]{8} & \multirow{2}{*}[-0.1em]{8} & Linear & 333.1 & 72.8 & 7.4 & 1.2 & 54.4 & 3.3 \\
 &  &  & Cubic & 39.5 & 13.2 & 2.6 & 0.5 & 23.1 & 1.4 \\
\cmidrule{3-10}
 &  & \multirow{2}{*}[-0.1em]{128} & Linear & 1252.8 & 138.4 & 7.9 & 1.6 & 48.4 & 10.0 \\
 &  &  & Cubic & 203.6 & 21.4 & 2.7 & 0.5 & 14.2 & 2.1 \\
\cmidrule{3-10}
 &  & \multirow{2}{*}[-0.1em]{2048} & Linear & 1838.2 & 164.6 & 7.9 & 1.6 & 47.8 & 11.8 \\
 &  &  & Cubic & 253.8 & 24.7 & 2.7 & 0.5 & 16.2 & 2.2 \\
\bottomrule
\end{tabular}
\end{table}

\begin{table}[htbp]
\centering
\caption{2D $\rightarrow$ 3D backward projection throughput in $10^3$ projections/second}
\label{tab:backproject-2d-to-3d-performance}
\begin{tabular}{r@{\hspace{3mm}} c@{\hspace{2mm}} r@{\hspace{3mm}} c@{\hspace{2mm}} | *{6}{>{\raggedleft\arraybackslash}p{1.5cm}}}
\toprule
\multirow{2}{*}[-0.6em]{Box} & \multirow{2}{*}[-0.6em]{Batch} & \multirow{2}{*}[-0.6em]{Poses} & \multirow{2}{*}[-0.6em]{Interp.} & \multicolumn{2}{c}{H100: CUDA} & \multicolumn{2}{c}{M4: CPU} & \multicolumn{2}{c}{M4: MPS} \\
& & & & Fwd & Fwd+Bwd & Fwd & Fwd+Bwd & Fwd & Fwd+Bwd \\
\midrule
\multirow{12}{*}[-1.2em]{32} & \multirow{6}{*}[-0.6em]{1} & \multirow{2}{*}[-0.1em]{8} & Linear & 127.9 & 25.1 & 39.2 & 20.9 & 15.9 & 7.3 \\
 &  &  & Cubic & 64.1 & 21.1 & 10.8 & 8.5 & 3.8 & 1.0 \\
\cmidrule{3-10}
 &  & \multirow{2}{*}[-0.1em]{128} & Linear & 1717.8 & 377.6 & 54.7 & 46.7 & 311.9 & 77.0 \\
 &  &  & Cubic & 521.9 & 256.7 & 12.9 & 12.1 & 77.5 & 55.9 \\
\cmidrule{3-10}
 &  & \multirow{2}{*}[-0.1em]{2048} & Linear & 4928.6 & 3077.0 & 57.0 & 53.1 & 779.8 & 599.4 \\
 &  &  & Cubic & 708.5 & 637.2 & 9.8 & 12.6 & 109.1 & 100.8 \\
\cmidrule{3-10}
\cmidrule{2-10}
 & \multirow{6}{*}[-0.6em]{8} & \multirow{2}{*}[-0.1em]{8} & Linear & 979.9 & 197.6 & 105.0 & 46.2 & 184.7 & 41.4 \\
 &  &  & Cubic & 424.7 & 153.1 & 28.5 & 20.0 & 60.5 & 39.2 \\
\cmidrule{3-10}
 &  & \multirow{2}{*}[-0.1em]{128} & Linear & 5228.2 & 2293.9 & 131.9 & 86.0 & 748.4 & 493.4 \\
 &  &  & Cubic & 887.5 & 701.9 & 28.4 & 24.5 & 114.3 & 102.4 \\
\cmidrule{3-10}
 &  & \multirow{2}{*}[-0.1em]{2048} & Linear & 5492.8 & 4665.0 & 124.7 & 72.9 & 935.4 & 783.9 \\
 &  &  & Cubic & 714.8 & 681.4 & 27.5 & 22.9 & 118.8 & 113.1 \\
\midrule
\multirow{12}{*}[-1.2em]{128} & \multirow{6}{*}[-0.6em]{1} & \multirow{2}{*}[-0.1em]{8} & Linear & 48.7 & 17.6 & 4.0 & 1.9 & 7.8 & 4.6 \\
 &  &  & Cubic & 8.6 & 5.9 & 1.4 & 0.8 & 1.8 & 1.5 \\
\cmidrule{3-10}
 &  & \multirow{2}{*}[-0.1em]{128} & Linear & 442.4 & 231.3 & 5.4 & 4.4 & 44.4 & 32.1 \\
 &  &  & Cubic & 64.9 & 52.6 & 1.5 & 1.4 & 6.2 & 5.6 \\
\cmidrule{3-10}
 &  & \multirow{2}{*}[-0.1em]{2048} & Linear & 570.3 & 473.0 & 5.8 & 5.0 & 56.8 & 46.6 \\
 &  &  & Cubic & 72.5 & 67.6 & 1.5 & 1.3 & 7.2 & 6.8 \\
\cmidrule{3-10}
\cmidrule{2-10}
 & \multirow{6}{*}[-0.6em]{8} & \multirow{2}{*}[-0.1em]{8} & Linear & 280.6 & 92.2 & 5.8 & 2.3 & 19.8 & 6.3 \\
 &  &  & Cubic & 53.3 & 34.5 & 1.7 & 1.1 & 4.7 & 3.1 \\
\cmidrule{3-10}
 &  & \multirow{2}{*}[-0.1em]{128} & Linear & 530.7 & 393.0 & 5.1 & 5.2 & 51.0 & 31.2 \\
 &  &  & Cubic & 70.6 & 64.8 & 1.7 & 1.4 & 7.0 & 6.3 \\
\cmidrule{3-10}
 &  & \multirow{2}{*}[-0.1em]{2048} & Linear & 587.7 & 508.2 & 6.9 & 5.4 & 57.3 & 41.8 \\
 &  &  & Cubic & 73.8 & 69.3 & 1.9 & 1.5 & 7.3 & 6.8 \\
\midrule
\multirow{12}{*}[-1.2em]{256} & \multirow{6}{*}[-0.6em]{1} & \multirow{2}{*}[-0.1em]{8} & Linear & 15.3 & 6.9 & 0.8 & 0.3 & 2.1 & 0.8 \\
 &  &  & Cubic & 2.3 & 1.7 & 0.3 & 0.2 & 0.5 & 0.3 \\
\cmidrule{3-10}
 &  & \multirow{2}{*}[-0.1em]{128} & Linear & 123.3 & 74.6 & 1.4 & 1.0 & 11.2 & 6.2 \\
 &  &  & Cubic & 17.6 & 14.7 & 0.4 & 0.3 & 1.6 & 1.4 \\
\cmidrule{3-10}
 &  & \multirow{2}{*}[-0.1em]{2048} & Linear & 135.8 & 114.8 & 1.5 & 1.1 & 14.3 & 10.0 \\
 &  &  & Cubic & 18.5 & 17.1 & 0.4 & 0.3 & 1.8 & 1.7 \\
\cmidrule{3-10}
\cmidrule{2-10}
 & \multirow{6}{*}[-0.6em]{8} & \multirow{2}{*}[-0.1em]{8} & Linear & 69.5 & 14.8 & 0.9 & 0.3 & 3.1 & 0.9 \\
 &  &  & Cubic & 13.8 & 7.3 & 0.4 & 0.2 & 1.1 & 0.6 \\
\cmidrule{3-10}
 &  & \multirow{2}{*}[-0.1em]{128} & Linear & 101.7 & 69.6 & 1.1 & 0.8 & 9.4 & 5.1 \\
 &  &  & Cubic & 16.1 & 14.3 & 0.4 & 0.3 & 1.8 & 1.4 \\
\cmidrule{3-10}
 &  & \multirow{2}{*}[-0.1em]{2048} & Linear & 134.0 & 113.9 & 1.5 & 1.1 & 10.4 & 7.0 \\
 &  &  & Cubic & 18.7 & 17.4 & 0.4 & 0.3 & 1.8 & 1.5 \\
\bottomrule
\end{tabular}
\end{table}

\subsection{Performance Comparison with torch-fourier-slice}

Direct comparison with the existing \texttt{torch-fourier-slice} library demonstrates substantial performance improvements (Tables~\ref{tab:project-3d-to-2d-comparison}, \ref{tab:backproject-2d-to-3d-comparison}). \texttt{torch-projectors} consistently outperforms \texttt{torch-fourier-slice} by 1-2 orders of magnitude across all tested configurations. The new operators also have a significantly lower memory footprint because they calculate everything in a single step and avoid storing intermediate tensors.

For \texttt{torch-fourier-slice}, most testing scenarios for forward projection could only be tested in the forward pass, as memory consumption in the backward pass exceeded the test system's 48\,GB of memory. One particularly memory-intensive backward projection scenario could not be tested in either pass.

The CUDA backend benefits most from replacing multiple kernel invocations with a single custom kernel, as this reduces GPU driver overhead and dramatically improves memory locality in these memory-bandwidth-limited algorithms. The performance gap is less pronounced on CPU and MPS, where the relative overhead of multiple kernel calls is smaller, and the ratio between ALUs and memory bandwidth is lower.

Only 3D->2D forward and 2D->3D backward projections were compared because \texttt{torch-fourier-slice} does not implement 2D->2D operations.

\begin{table}[htbp]
\centering
\caption{3D$\rightarrow$2D forward projection in torch-projectors and torch-fourier-slice, throughput in $10^3$ projections/second}
\label{tab:project-3d-to-2d-comparison}
\begin{tabular}{r@{\hspace{3mm}} c@{\hspace{2mm}} r@{\hspace{3mm}} | >{\raggedleft\arraybackslash}p{0.8cm} >{\raggedleft\arraybackslash}p{0.8cm} >{\raggedleft\arraybackslash}p{0.8cm} | >{\raggedleft\arraybackslash}p{0.8cm} >{\raggedleft\arraybackslash}p{0.8cm} >{\raggedleft\arraybackslash}p{0.8cm} | >{\raggedleft\arraybackslash}p{0.8cm} >{\raggedleft\arraybackslash}p{0.8cm} >{\raggedleft\arraybackslash}p{0.8cm} | >{\raggedleft\arraybackslash}p{0.8cm} >{\raggedleft\arraybackslash}p{0.8cm} >{\raggedleft\arraybackslash}p{0.8cm}}
\toprule
\multirow{3}{*}[-1.2em]{Box} & \multirow{3}{*}[-1.2em]{Batch} & \multirow{3}{*}[-1.2em]{Poses} & \multicolumn{6}{c}{H100: CUDA} & \multicolumn{6}{c}{M4: CPU} \\
& & & \multicolumn{3}{c}{Forward} & \multicolumn{3}{c}{Backward} & \multicolumn{3}{c}{Forward} & \multicolumn{3}{c}{Backward} \\
& & & t-f-s & t-p & $\uparrow$ & t-f-s & t-p & $\uparrow$ & t-f-s & t-p & $\uparrow$ & t-f-s & t-p & $\uparrow$ \\
\midrule
\multirow{6}{*}[-1.2em]{32} & \multirow{3}{*}[-0.6em]{1} & 8 & 5.4 & 28 & \textbf{$\times$5.3} & 6.2 & 42 & \textbf{$\times$6.7} & 9.1 & 153 & \textbf{$\times$17} & 0.3 & 55 & \textbf{$\times$202} \\
\cmidrule{3-15}
 &  & 128 & 84 & 449 & \textbf{$\times$5.3} & 18 & 621 & \textbf{$\times$35} & 38 & 643 & \textbf{$\times$17} & 0.2 & 87 & \textbf{$\times$364} \\
\cmidrule{3-15}
 &  & 2048 & 922 & 6604 & \textbf{$\times$7.2} & --- & 4174 & --- & 118 & 981 & \textbf{$\times$8.3} & --- & 109 & --- \\
\cmidrule{2-15}
 & \multirow{3}{*}[-0.6em]{8} & 8 & 6.0 & 1098 & \textbf{$\times$183} & 7.0 & 323 & \textbf{$\times$46} & 11 & 450 & \textbf{$\times$40} & 0.3 & 102 & \textbf{$\times$371} \\
\cmidrule{3-15}
 &  & 128 & 95 & 18894 & \textbf{$\times$200} & 18 & 4509 & \textbf{$\times$247} & 40 & 841 & \textbf{$\times$21} & 0.2 & 206 & \textbf{$\times$874} \\
\cmidrule{3-15}
 &  & 2048 & 941 & 85962 & \textbf{$\times$91} & --- & 7542 & --- & 115 & 935 & \textbf{$\times$8.2} & --- & 234 & --- \\
\midrule
\multirow{6}{*}[-1.2em]{128} & \multirow{3}{*}[-0.6em]{1} & 8 & 6.0 & 20 & \textbf{$\times$3.4} & --- & 36 & --- & 2.8 & 41 & \textbf{$\times$15} & --- & 7.3 & --- \\
\cmidrule{3-15}
 &  & 128 & 59 & 321 & \textbf{$\times$5.5} & --- & 410 & --- & 7.5 & 69 & \textbf{$\times$9.3} & --- & 10 & --- \\
\cmidrule{3-15}
 &  & 2048 & 99 & 2781 & \textbf{$\times$28} & --- & 728 & --- & 8.3 & 68 & \textbf{$\times$8.2} & --- & 10 & --- \\
\cmidrule{2-15}
 & \multirow{3}{*}[-0.6em]{8} & 8 & 6.5 & 759 & \textbf{$\times$118} & --- & 259 & --- & 2.7 & 44 & \textbf{$\times$17} & --- & 7.8 & --- \\
\cmidrule{3-15}
 &  & 128 & 60 & 5129 & \textbf{$\times$85} & --- & 663 & --- & 7.1 & 55 & \textbf{$\times$7.8} & --- & 11 & --- \\
\cmidrule{3-15}
 &  & 2048 & 99 & 7721 & \textbf{$\times$78} & --- & 761 & --- & 5.9 & 56 & \textbf{$\times$9.4} & --- & 11 & --- \\
\midrule
\multirow{6}{*}[-1.2em]{256} & \multirow{3}{*}[-0.6em]{1} & 8 & 5.5 & 13 & \textbf{$\times$2.3} & --- & 18 & --- & 1.0 & 9.5 & \textbf{$\times$9.1} & --- & 1.2 & --- \\
\cmidrule{3-15}
 &  & 128 & 22 & 206 & \textbf{$\times$9.3} & --- & 157 & --- & 1.8 & 10 & \textbf{$\times$5.9} & --- & 2.2 & --- \\
\cmidrule{3-15}
 &  & 2048 & 26 & 917 & \textbf{$\times$36} & --- & 191 & --- & 1.7 & 11 & \textbf{$\times$6.6} & --- & 2.3 & --- \\
\cmidrule{2-15}
 & \multirow{3}{*}[-0.6em]{8} & 8 & 5.7 & 314 & \textbf{$\times$55} & --- & 94 & --- & 1.0 & 7.0 & \textbf{$\times$7.0} & --- & 1.4 & --- \\
\cmidrule{3-15}
 &  & 128 & 22 & 1246 & \textbf{$\times$56} & --- & 166 & --- & 1.7 & 7.9 & \textbf{$\times$4.6} & --- & 2.0 & --- \\
\cmidrule{3-15}
 &  & 2048 & 26 & 1835 & \textbf{$\times$71} & --- & 196 & --- & 1.5 & 8.0 & \textbf{$\times$5.3} & --- & 2.0 & --- \\
\bottomrule
\end{tabular}
\end{table}

\begin{table}[htbp]
\centering
\caption{2D$\rightarrow$3D backward projection in torch-projectors and torch-fourier-slice, throughput in $10^3$ projections/second}
\label{tab:backproject-2d-to-3d-comparison}
\begin{tabular}{r@{\hspace{3mm}} c@{\hspace{2mm}} r@{\hspace{3mm}} | >{\raggedleft\arraybackslash}p{0.8cm} >{\raggedleft\arraybackslash}p{0.8cm} >{\raggedleft\arraybackslash}p{0.8cm} | >{\raggedleft\arraybackslash}p{0.8cm} >{\raggedleft\arraybackslash}p{0.8cm} >{\raggedleft\arraybackslash}p{0.8cm} | >{\raggedleft\arraybackslash}p{0.8cm} >{\raggedleft\arraybackslash}p{0.8cm} >{\raggedleft\arraybackslash}p{0.8cm} | >{\raggedleft\arraybackslash}p{0.8cm} >{\raggedleft\arraybackslash}p{0.8cm} >{\raggedleft\arraybackslash}p{0.8cm}}
\toprule
\multirow{3}{*}[-1.2em]{Box} & \multirow{3}{*}[-1.2em]{Batch} & \multirow{3}{*}[-1.2em]{Poses} & \multicolumn{6}{c}{H100: CUDA} & \multicolumn{6}{c}{M4: CPU} \\
& & & \multicolumn{3}{c}{Forward} & \multicolumn{3}{c}{Backward} & \multicolumn{3}{c}{Forward} & \multicolumn{3}{c}{Backward} \\
& & & t-f-s & t-p & $\uparrow$ & t-f-s & t-p & $\uparrow$ & t-f-s & t-p & $\uparrow$ & t-f-s & t-p & $\uparrow$ \\
\midrule
\multirow{6}{*}[-1.2em]{32} & \multirow{3}{*}[-0.6em]{1} & 8 & 3.3 & 97 & \textbf{$\times$29} & 10 & 33 & \textbf{$\times$3.2} & 6.5 & 45 & \textbf{$\times$6.8} & 21 & 53 & \textbf{$\times$2.5} \\
\cmidrule{3-15}
 &  & 128 & 47 & 1454 & \textbf{$\times$31} & 151 & 515 & \textbf{$\times$3.4} & 14 & 54 & \textbf{$\times$3.8} & 50 & 292 & \textbf{$\times$5.9} \\
\cmidrule{3-15}
 &  & 2048 & 159 & 4848 & \textbf{$\times$30} & 1222 & 8291 & \textbf{$\times$6.8} & 26 & 58 & \textbf{$\times$2.3} & 123 & 542 & \textbf{$\times$4.4} \\
\cmidrule{2-15}
 & \multirow{3}{*}[-0.6em]{8} & 8 & 3.6 & 754 & \textbf{$\times$211} & 14 & 262 & \textbf{$\times$19} & 6.5 & 96 & \textbf{$\times$15} & 21 & 89 & \textbf{$\times$4.2} \\
\cmidrule{3-15}
 &  & 128 & 51 & 5028 & \textbf{$\times$98} & 202 & 4251 & \textbf{$\times$21} & 13 & 126 & \textbf{$\times$9.6} & 41 & 362 & \textbf{$\times$8.8} \\
\cmidrule{3-15}
 &  & 2048 & 162 & 5614 & \textbf{$\times$35} & 1298 & 33718 & \textbf{$\times$26} & 25 & 129 & \textbf{$\times$5.2} & 102 & 539 & \textbf{$\times$5.3} \\
\midrule
\multirow{6}{*}[-1.2em]{128} & \multirow{3}{*}[-0.6em]{1} & 8 & 3.1 & 44 & \textbf{$\times$14} & 9.2 & 32 & \textbf{$\times$3.4} & 0.7 & 4.2 & \textbf{$\times$5.6} & 1.9 & 3.8 & \textbf{$\times$2.0} \\
\cmidrule{3-15}
 &  & 128 & 14 & 419 & \textbf{$\times$29} & 76 & 483 & \textbf{$\times$6.4} & 1.6 & 5.4 & \textbf{$\times$3.4} & 7.0 & 23 & \textbf{$\times$3.4} \\
\cmidrule{3-15}
 &  & 2048 & 18 & 557 & \textbf{$\times$31} & 108 & 2795 & \textbf{$\times$26} & 1.7 & 5.5 & \textbf{$\times$3.2} & 7.0 & 35 & \textbf{$\times$5.0} \\
\cmidrule{2-15}
 & \multirow{3}{*}[-0.6em]{8} & 8 & 3.4 & 261 & \textbf{$\times$77} & 12 & 146 & \textbf{$\times$12} & 0.7 & 4.8 & \textbf{$\times$6.7} & 1.6 & 3.7 & \textbf{$\times$2.3} \\
\cmidrule{3-15}
 &  & 128 & 15 & 529 & \textbf{$\times$36} & 80 & 1546 & \textbf{$\times$19} & 1.5 & 7.2 & \textbf{$\times$4.8} & 5.6 & 18 & \textbf{$\times$3.3} \\
\cmidrule{3-15}
 &  & 2048 & 18 & 588 & \textbf{$\times$33} & 102 & 3787 & \textbf{$\times$37} & 1.4 & 7.3 & \textbf{$\times$5.3} & 5.2 & 25 & \textbf{$\times$4.8} \\
\midrule
\multirow{6}{*}[-1.2em]{256} & \multirow{3}{*}[-0.6em]{1} & 8 & 2.1 & 15 & \textbf{$\times$7.1} & 6.5 & 13 & \textbf{$\times$2.0} & 0.2 & 0.8 & \textbf{$\times$3.7} & 0.4 & 0.5 & \textbf{$\times$1.4} \\
\cmidrule{3-15}
 &  & 128 & 4.1 & 124 & \textbf{$\times$30} & 24 & 198 & \textbf{$\times$8.3} & 0.3 & 1.4 & \textbf{$\times$4.5} & 1.3 & 3.2 & \textbf{$\times$2.4} \\
\cmidrule{3-15}
 &  & 2048 & 4.6 & 135 & \textbf{$\times$30} & 27 & 732 & \textbf{$\times$27} & 0.3 & 1.5 & \textbf{$\times$4.8} & 1.4 & 5.0 & \textbf{$\times$3.6} \\
\cmidrule{2-15}
 & \multirow{3}{*}[-0.6em]{8} & 8 & 2.3 & 70 & \textbf{$\times$30} & 7.3 & 27 & \textbf{$\times$3.7} & 0.2 & 0.9 & \textbf{$\times$4.3} & 0.3 & 0.5 & \textbf{$\times$1.5} \\
\cmidrule{3-15}
 &  & 128 & 4.1 & 102 & \textbf{$\times$25} & 23 & 285 & \textbf{$\times$12} & 0.3 & 1.3 & \textbf{$\times$4.6} & 1.2 & 2.8 & \textbf{$\times$2.4} \\
\cmidrule{3-15}
 &  & 2048 & 4.6 & 134 & \textbf{$\times$29} & --- & 791 & --- & 0.2 & 1.4 & \textbf{$\times$8.1} & --- & 4.0 & --- \\
\bottomrule
\end{tabular}
\end{table}

\subsection{Interpolation Efficiency}

Because interpolation happens in Fourier space, the most pronounced artifacts can be found in real space. This is the opposite of interpolating in real space, where Fourier-space artifacts are more pronounced.

Cubic interpolation without oversampling and linear interpolation with 2x oversampling result in roughly the same real-space signal attenuation up to ca. half the box size. In practice, most algorithms for EM data will restrict the real-space signal extent to this range to leave room for signal delocalization after CTF convolution. 

Three scenarios were benchmarked to provide guidance on interpolation scheme selection. The first scenario used 4096 128-pixel 2D references projected without padding using linear or cubic interpolation, with one pose per reference. The second scenario employed offline 2x padding, preparing 128-pixel projections from 256-pixel references. The third scenario matched the second but included the padding process (additional IFFT, padding, and FFT operations) in the timed code. This last scenario represents use cases where references are intermediate representations in longer processing chains and cannot be pre-padded offline. All scenarios timed forward and backward passes through the operator chain together.

Sampling on an oversampled grid incurs a noticeable performance penalty (ca. 10\% on M4 CPU, 25\% on M4 MPS, 35\% on H100 CUDA) due to increased cache misses when accessing larger amounts of reference data. This penalty decreases when fewer references are used. Using cubic interpolation without oversampling remains significantly slower than linear interpolation with offline oversampling, with the notable exception of the MPS platform.

The trade-off changes when oversampling must be achieved by padding references on-the-fly, which adds expensive Fourier transforms to the process. In this scenario, cubic interpolation without oversampling is ca. 4x faster than linear interpolation with on-the-fly 2x padding across all platforms.

\begin{table}[htbp]
\centering
\caption{Projecting 4096 128x128px references, throughput in $10^3$ projections/second}
\label{tab:oversampling-table}
\begin{tabular}{ll|ccc}
\toprule
Platform & Interpolation & Oversampling = 1.0 & Oversampling = 2.0 & Pad 2x, Oversampling = 2.0 \\
\midrule
\multirow{2}{*}{Mac, M4: CPU} & linear & 93.0 & 84.6 & 6.9 \\
& cubic & 34.5 & 33.6 & 6.2 \\
\midrule
\multirow{2}{*}{Mac, M4: MPS} & linear & 442.5 & 335.9 & 81.7 \\
& cubic & 332.7 & 308.6 & 80.3 \\
\bottomrule
\end{tabular}
\end{table}

\section{Discussion}
% TODO: Discussion of results and implications

\section{Availability}
The \texttt{torch-projectors} library is available as an open-source Python package at \url{https://github.com/warpem/torch-projectors} and can be installed via pip. For CUDA-enabled packages, please refer to the repository's README.

\bibliographystyle{plain}
\bibliography{references}

\end{document}